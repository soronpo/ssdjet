%%%%%%%%%%%%%%%%%%%%%%%%%%%%%%%%%%%%%%%%%%%%%%%%%%%%%%%%%%%%%%%%%%%%%%%%
\section{Introduction}
\label{sec:intro}
%%%%%%%%%%%%%%%%%%%%%%%%%%%%%%%%%%%%%%%%%%%%%%%%%%%%%%%%%%%%%%%%%%%%%%%%

Object detection is a major rule in many different devices, from mobile phones, cameras, and IoT devices, to drones, and autonomous cars. With the aid of machine learning, detecting an object also involves its classification, as shown in Figure \ref{fig:detection_example}.

Object detection algorithms work best on GPUs. GPUs are implemented with a large number of cores (\textit{streaming multiprocessors} (SMs)), therefore the high parallelism enables algorithms, such as object detection and machines learning, to run faster than on a CPU.

With increasing demand for low-energy modules, NVIDIA manufactures an embedded platform called Jetson. We have received the Jetson TX1 to explore the performance of different object detection algorithms. NVIDIA's Jetson TX1 is an embedded system-on-module (SoM) with quad-core ARM Cortex-A57, 4GB LPDDR4 and integrated 256-core Maxwell GPU. It is useful for deploying computer vision and deep learning in 10 watts of power.

In this paper we will discuss and analyze the performance of two algorithms: (1) \textit{Single Shot MultiBox Detector} (SSD) \cite{liu2016ssd}, as implemented with Caffe, and (2) \textit{You Only Look Once} (YOLO) \cite{redmon2016you}.

The remainder of this paper is organized as follows: Section \ref{sec:results} presents SSD and YOLO performance on Tegra TX1; Section \ref{sec:profiling} analyzes the algorithms execution using NVIDIA Visual Profiler; Section \ref{sec:installation} shares installation and execution tips we have gathered along the way, and we conclude in Section \ref{sec:conclusion}.

\begin{figure}[t]
	\includegraphics[width=0.48\textwidth]{./imgs/detection_example.png}
	\caption{Object detection example using YOLO}
	\label{fig:detection_example}
\end{figure}
